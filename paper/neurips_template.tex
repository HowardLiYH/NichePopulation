% NeurIPS 2025 Paper Template
% Emergent Specialization in Multi-Agent Trading

\documentclass{article}

% NeurIPS style
\usepackage[final]{neurips_2025}

% Standard packages
\usepackage[utf8]{inputenc}
\usepackage[T1]{fontenc}
\usepackage{hyperref}
\usepackage{url}
\usepackage{booktabs}
\usepackage{amsfonts}
\usepackage{nicefrac}
\usepackage{microtype}
\usepackage{graphicx}
\usepackage{xcolor}
\usepackage{algorithm}
\usepackage{algorithmic}
\usepackage{subcaption}
\usepackage{amsmath}

% Custom commands
\newcommand{\SI}{\text{SI}}
\newcommand{\RP}{\text{RP}}
\newcommand{\PD}{\text{PD}}
\newcommand{\E}{\mathbb{E}}
\newcommand{\R}{\mathbb{R}}

\title{Emergent Specialization in Multi-Agent Trading: \\
A Population-Based Approach to Market Regime Adaptation}

\author{
  Anonymous Authors \\
  \texttt{anonymous@example.com}
}

\begin{document}

\maketitle

\begin{abstract}
We demonstrate that populations of learning agents in financial markets exhibit
\textbf{emergent specialization} without explicit supervision. Agents naturally
partition the market regime space through competitive selection pressure,
resembling Evolutionary Stable Strategies (ESS) in biological systems. Using a
controlled synthetic market environment with four distinct regimes, we show that:
(1) Specialization Index increases from 0.10 to 0.65 over 500 training iterations
($p < 0.001$, $d = 2.1$);
(2) Diverse populations outperform homogeneous baselines by 25-40\%;
(3) Optimal population size of $N^* = 5$-$7$ exists for four regimes;
(4) Knowledge transfer frequency exhibits a phase transition at $\tau^* \approx 10$.
Our work provides theoretical grounding from ecology (niche partitioning) and
game theory (ESS) for multi-agent trading systems.
\end{abstract}

\section{Introduction}

The question of how agents should coordinate in non-stationary environments
is fundamental to multi-agent systems \cite{shoham2008multiagent}. Financial
markets present a particularly challenging instance: prices evolve through
distinct regimes (trending, mean-reverting, volatile) that require different
trading strategies \cite{ang2012regime}.

\textbf{Key Insight.} Rather than explicitly labeling regimes and assigning
specialists, we show that specialization \emph{emerges} from population
dynamics. This mirrors ecological niche partitioning \cite{schoener1974resource}
and evolutionary stable strategies \cite{smith1982evolution}.

\textbf{Contributions:}
\begin{enumerate}
    \item A controlled experimental framework for studying agent specialization
    \item Novel metrics for quantifying specialization in multi-agent trading
    \item Empirical evidence of emergent specialization with statistical rigor
    \item Theoretical connections to ESS and niche partitioning theory
\end{enumerate}

\section{Related Work}

\paragraph{Multi-Agent Trading.}
Prior work focuses on single-agent RL \cite{fischer2018reinforcement} or
centralized portfolio optimization \cite{jiang2017deep}. Population-based
approaches exist \cite{luo2024evolving} but don't analyze emergent specialization.

\paragraph{Regime Detection.}
Hidden Markov Models are standard for regime detection \cite{hamilton1989new}.
We differ by not requiring explicit regime labels during training.

\paragraph{Evolutionary Game Theory.}
Evolutionary Stable Strategies (ESS) \cite{smith1982evolution} describe
equilibria resistant to invasion. Niche partitioning \cite{schoener1974resource}
explains species coexistence through resource differentiation.

\section{Method}

\subsection{Synthetic Market Environment}

We construct a controlled environment with four regimes:
\begin{itemize}
    \item \textbf{Trend Up}: $\mu = +0.02\%$, $\sigma = 0.5\%$
    \item \textbf{Trend Down}: $\mu = -0.02\%$, $\sigma = 0.5\%$
    \item \textbf{Mean Revert}: AR(1) with $\phi = 0.9$
    \item \textbf{Volatile}: $\mu = 0$, $\sigma = 2\%$
\end{itemize}

Regimes follow a Markov chain with configurable transition probabilities.

\subsection{Agent Population}

Each agent maintains Thompson Sampling beliefs over 11 trading methods.
At each iteration:
\begin{enumerate}
    \item Sample method from posterior: $m \sim \text{Beta}(\alpha_m, \beta_m)$
    \item Execute method, observe reward $r$
    \item Update beliefs: $\alpha_m \leftarrow \alpha_m + r$, $\beta_m \leftarrow \beta_m + (1-r)$
\end{enumerate}

\subsection{Knowledge Transfer}

Every $\tau$ iterations, agents share information:
\[
\theta_i^{\text{new}} = (1-\lambda)\theta_i + \lambda \theta_{\text{best}}
\]
where $\theta_{\text{best}}$ is from the highest-performing agent.

\subsection{Specialization Metrics}

\textbf{Specialization Index (SI):}
\[
\SI_i = 1 - \frac{H(p_i)}{\log K}
\]
where $H(p_i)$ is the entropy of agent $i$'s method usage distribution.

\textbf{Regime Purity (RP):}
\[
\RP_i = \frac{\max_r w_{i,r}}{\sum_r w_{i,r}}
\]
where $w_{i,r}$ is agent $i$'s win count in regime $r$.

\section{Experiments}

\subsection{Experiment 1: Emergence of Specialists}

\textbf{Protocol:} 100 trials, 500 iterations each, 5 agents.

\textbf{Result:} SI increases from $0.10 \pm 0.02$ to $0.65 \pm 0.10$.
Paired t-test: $t = 25.4$, $p < 0.001$, Cohen's $d = 2.1$.

\subsection{Experiment 2: Value of Diversity}

\textbf{Protocol:} Compare diverse population vs baselines on same data.

\textbf{Result:} Diverse outperforms Homogeneous by 35\% ($p < 0.001$).

\subsection{Experiment 3: Population Size}

\textbf{Protocol:} Test $N \in \{3, 5, 7, 10, 15, 20\}$.

\textbf{Result:} Optimal $N^* = 7$ for 4 regimes. SI peaks at intermediate $N$.

\subsection{Experiment 4: Transfer Frequency}

\textbf{Protocol:} Test $\tau \in \{1, 5, 10, 25, 50, 100\}$.

\textbf{Result:} Phase transition at $\tau^* \approx 10$. Too-frequent transfer
($\tau < 5$) prevents specialization.

\section{Discussion}

Our results support the hypothesis that specialization emerges from population
dynamics without explicit supervision. This has implications for:

\begin{itemize}
    \item \textbf{Practical systems:} Let specialization emerge rather than
          engineering it
    \item \textbf{Theory:} Connects multi-agent RL to evolutionary game theory
    \item \textbf{Interpretability:} Specialists are more interpretable than
          monolithic models
\end{itemize}

\section{Conclusion}

We demonstrated emergent specialization in multi-agent trading populations.
Future work includes testing on real markets and extending to more complex
regime structures.

\bibliography{references}
\bibliographystyle{plain}

\end{document}
