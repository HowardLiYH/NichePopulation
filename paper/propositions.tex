% Theoretical Propositions for Emergent Specialization

\section{Theoretical Foundation}
\label{sec:theory}

We provide three theoretical propositions grounding our empirical findings
in ecological and game-theoretic principles.

\subsection{Proposition 1: Competitive Exclusion Principle}

\begin{proposition}[Competitive Exclusion]
In a multi-agent system where $n$ agents compete for rewards across $k$ regimes,
with competition intensity $c > 0$, if agents share identical strategies,
at most one agent survives per regime in equilibrium.
\end{proposition}

\begin{proof}[Proof Sketch]
Consider two agents $i, j$ with identical strategy $\sigma$ competing in regime $r$.
Let $\pi_i(r)$ and $\pi_j(r)$ be their expected payoffs.

If $\sigma_i = \sigma_j$, then in expectation:
\begin{equation}
\pi_i(r) = \pi_j(r) = \frac{V(r) - c}{2}
\end{equation}
where $V(r)$ is the regime value and $c$ is the competition cost.

For $c > 0$, any perturbation $\epsilon$ giving agent $i$ a slight advantage
leads to $\pi_i(r) > \pi_j(r)$, causing agent $j$'s reward to decrease.
This creates evolutionary pressure for differentiation.

By the competitive exclusion principle from ecology \cite{hardin1960}, 
complete competitors cannot coexist. Agents must specialize to survive.
\end{proof}

\subsection{Proposition 2: Specialization Lower Bound}

\begin{proposition}[SI Lower Bound Under Competition]
For a population of $n$ agents competing across $k$ heterogeneous regimes
with niche bonus $\lambda > 0$, the expected Specialization Index satisfies:
\begin{equation}
\mathbb{E}[SI] \geq \frac{\lambda}{1 + \lambda} \cdot \left(1 - \frac{1}{k}\right)
\end{equation}
\end{proposition}

\begin{proof}[Proof Sketch]
Let $p_i^{(a)}$ be agent $a$'s probability of selecting regime $i$.
The agent's expected reward includes the niche bonus:
\begin{equation}
R(a) = \sum_{i=1}^{k} p_i^{(a)} \cdot r_i + \lambda \cdot SI(a)
\end{equation}

To maximize reward, agents trade off base reward $r_i$ against 
specialization bonus $\lambda \cdot SI$.

At equilibrium, the marginal benefit of specialization equals
the marginal cost of forgoing other regimes:
\begin{equation}
\frac{\partial SI}{\partial p_i} \cdot \lambda = \frac{\partial}{\partial p_i}\left(\sum_{j \neq i} p_j \cdot r_j\right)
\end{equation}

Solving this optimization with $SI = 1 - H(p)/\log(k)$ yields
the lower bound through Lagrangian analysis.
\end{proof}

\subsection{Proposition 3: Mono-Regime Collapse}

\begin{proposition}[Mono-Regime Collapse]
When the environment contains only one regime ($k=1$), 
the expected Specialization Index approaches zero:
\begin{equation}
\lim_{k \to 1} \mathbb{E}[SI] = 0
\end{equation}
regardless of competition intensity or niche bonus.
\end{proposition}

\begin{proof}[Proof Sketch]
When $k=1$, all agents face identical conditions.
The entropy of any agent's regime distribution is:
\begin{equation}
H(p) = -\sum_{i=1}^{1} p_i \log p_i = -1 \cdot \log(1) = 0
\end{equation}

Therefore:
\begin{equation}
SI = 1 - \frac{H(p)}{\log(k)} = 1 - \frac{0}{\log(1)} = 1 - \frac{0}{0}
\end{equation}

By L'Hôpital's rule or direct analysis, this limit is 0.
Intuitively, with only one regime, there is nothing to specialize in.

This proposition explains why our Weather domain shows lower SI (0.205):
its regime distribution is dominated by ``stable'' (62\%), 
approaching a mono-regime environment.
\end{proof}

\subsection{Empirical Validation}

Our experiments validate all three propositions:

\begin{itemize}
\item \textbf{P1 (Competitive Exclusion)}: Agents develop distinct niche preferences,
with mean niche coverage of 70-100\% across domains (Table~\ref{tab:results}).

\item \textbf{P2 (SI Lower Bound)}: With $\lambda=0.3$ and $k=4$, the bound predicts
$SI \geq 0.173$. Our observed SI (0.20--0.45) exceeds this across all domains.

\item \textbf{P3 (Mono-Regime Collapse)}: Weather domain (62\% ``stable'') shows
significantly lower SI (0.205) compared to balanced domains like Commodities (0.411).
\end{itemize}

