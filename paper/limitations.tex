% Limitations Section for NeurIPS Paper

\section{Limitations}
\label{sec:limitations}

\paragraph{Domain Coverage.}
While we validate emergent specialization across four diverse domains 
(Crypto, Commodities, Weather, Solar), these represent a subset of 
possible applications. Extension to other domains such as healthcare, 
traffic management, or natural language processing remains future work.
Network restrictions prevented us from accessing USGS water data and 
EIA energy data, limiting our validation to four domains rather than six.

\paragraph{Regime Detection.}
Our regime classification relies on domain-specific heuristics 
(e.g., MA crossover for finance, temperature thresholds for weather).
While effective, these are not universally optimal. Learned regime 
detection could potentially improve results, but at the cost of 
additional complexity and potential overfitting.

\paragraph{Simplified Competition Model.}
Our competition mechanism uses winner-take-all dynamics with a 
fixed niche bonus $\lambda=0.3$. Real-world multi-agent systems 
may exhibit more complex reward structures, partial information, 
or dynamic competition intensity. Our theoretical analysis assumes 
static environments; extending to non-stationary settings is future work.

\paragraph{Computational Efficiency.}
While our approach is computationally efficient compared to deep 
MARL methods, the iterative learning process still requires 
significant computation for large-scale applications. We did not 
extensively optimize hyperparameters due to time constraints; 
additional tuning may yield improved results.

\paragraph{Baseline Scope.}
Our MARL comparison focuses on Independent Q-Learning (IQL) 
as the primary baseline. While we implemented QMIX and MAPPO, 
the simplified task structure may not fully leverage their 
coordination capabilities. More sophisticated environments 
with explicit coordination requirements could better showcase 
differences between approaches.

\paragraph{Generalization to Continuous Actions.}
Our framework assumes discrete method/strategy selection. 
Extension to continuous action spaces (e.g., portfolio weights) 
would require additional mechanism design to maintain the 
niche-based competition structure.

\paragraph{Statistical Limitations.}
With 30 trials per domain and 4 domains, some statistical tests 
(particularly cross-domain comparisons) have limited power. 
Larger-scale experiments with more trials could provide stronger 
statistical evidence, particularly for the Weather domain where 
improvement over baseline is marginal (+6.4\%).

